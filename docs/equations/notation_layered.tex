\documentclass[11pt]{article}
\usepackage{geometry}                
\geometry{a4paper,left=2.5cm,right=2.5cm,top=2.5cm,bottom=2.5cm}
\usepackage{natbib}
\usepackage{color}
\definecolor{mygreen}{RGB}{28,172,0} % color values Red, Green, Blue
\definecolor{mylilas}{RGB}{170,55,241}
\usepackage{epsfig}
\usepackage{amssymb,amsmath}
\usepackage{enumerate}
\usepackage{enumitem}
\usepackage[utf8]{inputenc}
\usepackage{hyperref}
\usepackage{mathtools}


\newcommand{\ssd}{\text{ssd}}
\newcommand{\sS}{\mathsf{S}}
\newcommand{\tot}{\text{tot}}

\begin{document}


\input{symbols}

\section*{Layered quasigeostrophic model}

The $\nmax$-layer quasigeostrophic (QG) potential vorticity is

\begin{align}
{q_1} &= \lap\psi_1 + \frac{f_0^2}{H_1} \left(\frac{\psi_{2}-\psi_1}{g'_{1}}\right)\,,  \qquad & i =1\com \nonumber \\
{q_n} &= \lap\psi_n + \frac{f_0^2}{H_n} \left(\frac{\psi_{n-1}-\psi_n}{g'_{n-1}}  - \frac{\psi_{i}-\psi_{n+1}}{g'_{i}}\right)\,,  \qquad &i = 2,\nmax-1 \com \nonumber \\
{q_\nmax} &= \lap\psi_\nmax + \frac{f_0^2}{H_\nmax} \left(\frac{\psi_{\textsf{N}-1}-\psi_\nmax}{g'_{\nmax-1}}\right) + \frac{f_0}{H_\nmax}h_b (x,y)\,,  \qquad & i =\nmax\,,
\end{align}
where $q_n$ is the n'th layer QG potential vorticity, and $\psi_n$ is the streamfunction, 
 $f_0$ is the inertial frequency, n'th $H_n$ is the layer depth, and $h_b$ is the 
bottom topography. (Note that in QG $h_b/H_\nmax << 1$.) Also the n'th buoyancy
jump (reduced gravity) is
\begin{equation}
g'_n \equiv g \frac{\rho_{n}-\rho_{n+1}}{\rho_n}\com
\end{equation}
where $g$ is the acceleration due to gravity and $\rho_n$ is the layer density.

The dynamics of the system is given by the evolution of PV. In particular, assuming a background
flow with background velocity $\vec{V} = (U,V)$ such that
\begin{align}
\label{eq:Uequiv}
u_n^{{\tot}} = U_n - \psi_{n y}\com \nonumber \\
v_n^{\tot} = V_n + \psi_{n x} \com
\end{align}
and
\begin{equation}
q_n^{\tot} = Q_n + \delta_{n\nmax}\frac{f_0}{H_\nmax}h_b + q_n \com
\end{equation}
where $Q_n + \delta_{n\nmax}\frac{f_0}{H_\nmax}h_b$ is n'th layer background PV,
we obtain the evolution equations
\begin{align}
\label{eq:qg_dynamics}
{q_n}_t + \mathsf{J}(\psi_n,q_n + \delta_{n \nmax} \frac{f_0}{H_\nmax}h_b )& + U_n ({q_n}_x + \delta_{n \nmax} \frac{f_0}{H_\nmax}h_{bx}) + V_n ({q_n}_y + \delta_{n \nmax} \frac{f_0}{H_\nmax}h_{by})+ \nonumber
\\ & {Q_n}_y {\psi_n}_x - {Q_n}_y {\psi_n}_y = \ssd
- r_{ek} \delta_{n\nmax} \lap \psi_n \com \qquad n = 1,\nmax\com
\end{align}
where $\ssd$ is 
stands for small scale dissipation, which is achieved by an spectral exponential filter
or hyperviscosity, and $r_{ek}$ is the linear bottom drag coefficient. The Dirac delta,
$\delta_{nN}$, indicates that the drag is only applied in the bottom layer.

\subsection*{Equations in spectral space}

The evolutionary equation in spectral space is

\begin{align}
    \hat{q}_{nt} + (\mathrm{i} k U + \mathrm{i} l V) \left(\hat{q}_n + \delta_{n \nmax} \frac{f_0}{H_\nmax}\hat{h}_b\right) + (\mathrm{i} k\, {Q_y} -  - \mathrm{i} l\,{Q_x}){\hat{\psi}_n} + \mathsf{\hat{J}}(\psi_n, q_n + \delta_{n \nmax} \frac{f_0}{H_\nmax}h_b )   \nonumber \\ =  \ssd + \mathrm{i}  \delta_{n \nmax} r_{ek} \kappa^2 \hat{\psi}_n \,, \qquad i = 1,\textsf{N}\com
\end{align}
where $\kappa^2 = k^2 + l^2$. Also, in the pseudo-spectral spirit we write the transform of the nonlinear
terms and the non-constant coefficient linear term as the transform of the products, calculated in physical space, as opposed to double convolution sums.  That is $\mathsf{\hat{J}}$ is the Fourier transform of Jacobian computed in physical space.

The inversion relationship is

\begin{equation}
    \hat{q}_i = {\left(\sS - \kappa^2 \sI \right)} \hat{\psi}_i\com
\end{equation}
where $\sI$ is the $\nmax\times\nmax$ identity matrix, and the stretching matrix is

\begin{equation}
\textsf{S} \equiv  f_0^2
\begin{bmatrix}
    -\frac{1}{g'_1 H_1}& & \frac{1}{g'_1 H_1} &  & 0 \dots& \\
 & 0 & & & & &\\
    \vdots & \ddots& &\ddots &\ddots & & & &\\
       & \frac{1}{g'_{i-1} H_i}& &  -\left(\frac{1}{g'_{i-1} H_i} + \frac{1}{g'_{i} H_i}\right)& & \frac{1}{g'_{i} H_i}\,\,\,\,\,\,\, \\
       & \ddots& & \ddots &\ddots & & & &\\
& & & & & \\
& \dots & 0 & \frac{1}{ g'_{\nmax-1} H_\nmax}& & -\frac{1}{g'_{\nmax-1} H_\nmax}
\end{bmatrix}
\per
\end{equation}

\subsection*{Energy spectrum}
The equation for the energy spectrum,
\begin{equation}
E(k,l) \equiv {\frac{1}{2 H}\sum_{i=1}^{\nmax} H_i \kappa^2 |\hat{\psi}_i|^2} \,\,\,\,+ \,\,\,\,\,\, {\frac{1}{2 H} \sum_{i=1}^{\nmax-1} \frac{f_0^2}{g'_i}|\hat{\psi}_{i}- \hat{\psi}_{i+1}|^2}\,\,\,\,,
\end{equation}
is 

\begin{align}
    \frac{d}{dt} E(k,l) = {\frac{1}{H}\sum_{i=1}^{\mathsf{N}} H_i \text{Re}[\hat{\psi}_i^\star {\mathsf{\hat{J}}}(\psi_i,\nabla^2\psi_i)]} +
    {\frac{1}{H}\sum_{i=1}^{\mathsf{N}} H_i\text{Re}[\hat{\psi}_i^\star \hat{\mathsf{J} (\psi_i,(\sS \psi)_i)}]} \nonumber \\
    + {\frac{1}{H}\sum_{i=1}^{\mathsf{N}} H_i ( k U_i +  l V_i)\, \text{Re}[i \, \hat{\psi}^\star_i (\mathsf{S}\hat{\psi}_i)]} \,\,\,\,\,\,\,{- r_{ek} \frac{H_\mathsf{N}}{H} \kappa^2 |\hat{\psi}_{\mathsf{N}}|^2}  +{ {{E_\ssd}}} \com
\end{align}
where $\star$ stands for complex conjugation, and the terms above on the right represent, from
left to right,

\begin{description}
    \item[I:]  The spectral divergence of the kinetic energy flux;
    \item[II:] The spectral divergence of the potential energy flux; 
    \item[III:] The spectrum of the potential energy generation;
    \item[IV:] The spectrum of the energy dissipation by linear bottom drag;
    \item[V:] The spectrum of energy loss due to small scale dissipation.
\end{description}
We assume that $V$ is relatively small, and that, in statistical steady state, the
budget above is dominated by I through IV.

\subsection*{Enstrophy spectrum}
Similarly the evolution of the barotropic enstrophy spectrum,

\begin{equation}
Z(k,l) \equiv \frac{1}{2H} \sum_{i=1}^{\nmax} H_i |\hat{q}_i|^2\com
\end{equation}
is governed by
\begin{equation}
    \frac{d}{d t} Z(k,l) = {\text{Re}[\hat{q}_i^\star {\mathsf{\hat{J}}(\psi_i,q_i) ]}}
    {-(k Q_y - l Q_x)\text{Re}[(\sS \hat{\psi}_i^\star)\hat{\psi}_i]}
    + { {\hat{Z_\ssd}}}\com
\end{equation}
where the terms above on the right represent, from
left to right,
\begin{description}
    \item[I:]   The spectral divergence of barotropic potential enstrophy flux;
    \item[II:]  The spectrum of  barotropic potential enstrophy generation;
    \item[III:] The spectrum of  barotropic potential enstrophy loss due to small scale dissipation.
\end{description}
The enstrophy dissipation is concentrated at the smallest scales resolved in the model and, in statistical steady
state, we expect the budget above to be dominated by the balance between I and mathrm{i}.

\subsection*{Vertical modes}
Standard vertical modes, , $\sp_n (z)$, are the eigenvectors of the ``stretching matrix''

\begin{equation}
\sS \,\sp_n = -R_n^{-2}\, \sp_n\,,
\end{equation}
where the $R_n$ is the n'th deformation radius.

\subsection*{Linear stability analysis}

With $h_b = 0$, the linear eigenproblem is

\begin{equation}
 \sA \, \mathsf{\Phi} = \omega \, \sB\, \mathsf{\Phi}\,,
\end{equation}
where

\begin{equation}
\sA \equiv \mathsf{B}(\mathsf{U}\, k + \mathsf{V}\,l) + \mathsf{I}\left(k\,\mathsf{Q}_y - l\,\mathsf{Q}_x\right) + \mathsf{I}\,\delta_{\mathsf{N}\mathsf{N}}\, \mathrm{i} \,r_{ek}\,\kappa^2\,,
\end{equation}
where $\delta_{\mathsf{N}\mathsf{N}} = [0,0,\dots,0,1]\,,$
and 
\begin{equation}
\mathsf{B} \equiv  \mathsf{S} - \mathsf{I} \kappa^2\,. 
\end{equation}
The growth rate is Im$\{\omega\}$.

\section*{Special case: two-layer model}
With $\nmax = 2$, an alternative notation for the perturbation of potential vorticities can be written as
\begin{align}
    q_1 &= \lap \psi_1 + F_1 (\psi_2 - \psi_1) \nonumber\\
    q_2 &= \lap \psi_2 + F_2 (\psi_1  - \psi_2)\com
\end{align}
where we use the following definitions
where
\begin{equation}
F_1 \equiv \frac{k_d^2}{1 + \delta^2}\,, \qquad \:\:\text{and} \qquad F_2 \defn \delta \,F_1\,,
\end{equation}
with the deformation wavenumber
\begin{equation}
k_d^2 \equiv \, \frac{f_0^2}{g} \frac{H_1+H_2}{H_1 H_2} \per
\end{equation}
With this notation, the ``stretching matrix'' is simply
\begin{equation}
\sS = \begin{bmatrix}
- F_1 \qquad \:\:\:\:F_1\\
F_2 \qquad -  + F_2
\end{bmatrix}\per
\end{equation}
The inversion relationship in Fourier space is
\begin{equation}
\begin{bmatrix}
\hat{\psi}_1\\
\hat{\psi}_2\\
\end{bmatrix}
= \frac{1}{\text{det} \: \sB}
\begin{bmatrix}
-(\kappa^2 + F_2) \qquad \:\:\:\:-F_1\\
\:\:\:\: -F_2 \qquad - (\kappa^2 + F_1)
\end{bmatrix}
\begin{bmatrix}
\hat{q}_1\\
\hat{q}_2\\
\end{bmatrix}\com
\end{equation}
where 
\begin{equation}
\qquad \text{det}\, \sB = \kappa^2\left(\kappa^2 + F_1 + F_2\right)\,.
\end{equation}

\end{document}

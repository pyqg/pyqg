\documentclass[11pt]{article}
\usepackage{geometry}                
\geometry{a4paper,left=2.5cm,right=2.5cm,top=2.5cm,bottom=2.5cm}
\usepackage{natbib}
\usepackage{color}
\definecolor{mygreen}{RGB}{28,172,0} % color values Red, Green, Blue
\definecolor{mylilas}{RGB}{170,55,241}
\usepackage{epsfig}
\usepackage{amssymb,amsmath}
\usepackage{enumerate}
\usepackage{enumitem}
\usepackage[utf8]{inputenc}
\usepackage{hyperref}
\usepackage{mathtools}

\newcommand{\ssd}{\text{ssd}}
\newcommand{\sS}{\mathsf{S}}
\newcommand{\tot}{\text{tot}}

\usepackage{hyperref}


\begin{document}


\input{symbols}

\section*{Vertical modes}

Standard vertical modes, , $\sp_n (z)$, are the eigenvectors of the ``stretching matrix''

\begin{equation}
\sS \,\sp_n = -R_n^{-2}\, \sp_n\,,
\end{equation}
where the $R_n$ is by definition the n'th deformation radius (e.g., \href{http://www.sciencedirect.com/science/article/pii/0377026578900027}{Flierl 1978}). These orthogonal modes $\sp_n$ are normalized to
have unitary $L2$-norm
\begin{equation}
    \frac{1}{H}\int_{-H}^{0} \sp_n \sp_m \dd z = \delta_{nm}\com
\end{equation}
where $\delta_{mn}$.

\end{document}
